\documentclass{jsarticle}
\usepackage{amsmath,amssymb,amsfonts}
\usepackage{amsthm}
\theoremstyle{definition}
\newtheorem{theorem}{定理}
\newtheorem*{theorem*}{定理}
\newtheorem{definition}{定義}
\newtheorem*{definition*}{定義}
\begin{document}
\title{逆関数定理と陰関数定理}
\author{む}
\date{\today}
\maketitle
\begin{abstract}
ここでは、多様体にこれから入門する人を対象として、多様体に入門するにあたって重要な定理である逆関数定理、陰関数定理について述べる。

ただし、これらの定理の証明に関しては一切述べず、その代わり定理の意味する所を、具体的な関数を例に、図も用いて解説していく。
\end{abstract}

\tableofcontents

\section{2つの定理の主張とその需要}

\begin{theorem}[逆関数定理]
$m\ge 0$、$1\le r\le \infty$、$p\in U\overset{\text{open}}{\subset} \mathbb{R}^m$とする。

$f : U\rightarrow \mathbb{R}^m$:$C^r$級関数が点$p$において$(Jf)_p$: 正則を満たす時、
\end{theorem}

\section{逆関数定理}
ユークリッド空間の部分集合に多様体として整合的な局所的構造を与えるための
\section{陰関数定理}

\end{document}